% \documentclass{beamer}
\documentclass[xcolor=table]{beamer}
\usepackage{hyperref}
\usepackage{parskip}
\usepackage{verbatim}
\usepackage{tabularx}
\usepackage{colortbl}
% \usetheme{Warsaw}
\usetheme{Boadilla}
\usepackage{Sweave}
\begin{document}
\Sconcordance{concordance:bios545r_latex_week9_graphics.tex:bios545r_latex_week9_graphics.Rnw:%
1 145 1}

% \input{datatable-concordance}

% Slide 01 - Title Slide

\title{Introduction to Graphics}
\subtitle{Steve Pittard wsp@emory.edu}

\maketitle



% Slide 02

\begin{frame}[fragile]
\frametitle{Introduction to Graphics}
\begin{itemize}
\item R has a poweful environment for visualization of scientific data
\item It offers publication quality graphics which can be programmed
\item Easily reproducibe across platforms (OSX, Windows, Linux)
\item Full LaTeX and Sweave support
\item Many add-on packages and functions with built-in graphics support 
\item Can easily save to PDF, JPEG, PNG, and SVG
\end{itemize}
\end{frame}

%

\begin{frame}[fragile]
\frametitle{Some sources for Information}
\begin{itemize}
\item R graph gallery \url{http://gallery.r-enthusiasts.com/}
\item R Graphical Manual \url{http://bit.ly/1ULffYH}
\item Lattice Graphics \url{http://bit.ly/1Od8fOQ}
\item ggplot2 \url{http://ggplot2.org/}
\end{itemize}
\end{frame}

%

\begin{frame}[fragile]
\frametitle{History}
R graphics can be confusing because there are four different systems that get mentioned in textbooks and online manuals for R. 

I will list them here and talk about briefly about each.

\begin{itemize}
\item Low Level Capability
\begin{itemize}
\item Base Graphics (Has Low and High Level functions)
\item Grid Graphics (We don't get into this package)
\end{itemize}
\item High Level Capability
\begin{itemize}
\item{Lattice Graphics}
\item{ggplot2}
\end{itemize}
\end{itemize}
\end{frame}

%

\begin{frame}[fragile]
\frametitle{Base Graphics}
\begin{itemize}
\item Oldest and most commonly used (although that might be changing)
\item Uses a ``pen on paper'' model
\begin{itemize}
 \item You can only draw on top of what has already been drawn/plotted
 \item Cannot erase, modify, or move existing plot objects
 \item Easy to start over if you make mistakes
\end{itemize}

\item Has both high and low level plotting routines
\item Very fast
\item Can build complex plots in steps 
\item Can easily write functions to do plotting 
\item Lots of documentation and ``Google'' support
\end{itemize}

\end{frame}

%

\begin{frame}[fragile]
\frametitle{Grid Graphics}
\begin{itemize}
\item Developed in 2000 by Paul Murrell
\item Provides a rich set of graphics primitive
\item Uses a system of objects and view ports to make plotting complex objects easier
\item You almost never use this directly unless you want to do low level programming (mostly true)
\item Both the \textbf{lattice} and \textbf{ggplot2} packages use Grid
\item We will not be learning about this package in this class beyond this slide
\item Manual is at \url{https://stat.ethz.ch/R-manual/R-devel/library/grid/doc/grid.pdf}
\end{itemize}

\end{frame}

%

\begin{frame}[fragile]
\frametitle{Lattice Graphics}
\begin{itemize}
\item Developed by Deepayan Sarkar 
\item Insipred by Trellis graphics and the book "Visualizing Data" by William Cleveland
\item Easy to create conditioned plots and panelled plots (aka Trellis plots)
\item Easy to do grouping of data within a single panel 
\item Legends, annotations, and axes are automatically created
\item Is a separate package but it comes with any installation of R
\item See \url{http://lmdvr.r-forge.r-project.org/figures/figures.html}
\end{itemize}
\footnotesize
\begin{verbatim}
library(lattice)
xyplot(mpg ~ wt | factor(am, labels=c("Auto","Manual)),data = mtcars)
\end{verbatim}

\end{frame}

%

%

\begin{frame}[fragile]
\frametitle{Lattice Graphics}
\begin{center}
\includegraphics{xyplot.png}
\end{center}
\end{frame}


%%
%%
%% End of Doc
\end{document}
